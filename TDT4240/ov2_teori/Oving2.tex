%
%  untitled
%
%  Created by Jim Frode Hoff on 2014-02-07.
%  Copyright (c) 2014 . All rights reserved.
%
\documentclass[]{article}

% Use utf-8 encoding for foreign characters
\usepackage[utf8]{inputenc}

% Setup for fullpage use
\usepackage{fullpage}

% Uncomment some of the following if you use the features
%
% Running Headers and footers
%\usepackage{fancyhdr}

% Multipart figures
%\usepackage{subfigure}

% More symbols
%\usepackage{amsmath}
%\usepackage{amssymb}
%\usepackage{latexsym}

% Surround parts of graphics with box
\usepackage{boxedminipage}

% Package for including code in the document
\usepackage{listings}

% If you want to generate a toc for each chapter (use with book)
\usepackage{minitoc}

% This is now the recommended way for checking for PDFLaTeX:
\usepackage{ifpdf}

%\newif\ifpdf
%\ifx\pdfoutput\undefined
%\pdffalse % we are not running PDFLaTeX
%\else
%\pdfoutput=1 % we are running PDFLaTeX
%\pdftrue
%\fi

\ifpdf
\usepackage[pdftex]{graphicx}
\else
\usepackage{graphicx}
\fi
\title{TDT4240 - Excerise 2}
\author{Linn Vikre\\ Jim Frode Hoff}

\date{2014-02-07}

\begin{document}

\ifpdf
\DeclareGraphicsExtensions{.pdf, .jpg, .tif}
\else
\DeclareGraphicsExtensions{.eps, .jpg}
\fi

\maketitle

\section{Introduction}
\emph{3.a) For the patterns listing in Step3, which are architectural patterns 
and which are design patterns? What are the relationships and differences of 
architectural patterns and design patterns?}
\\

\begin{tabular}{ | l | l | } \hline
  Observer & 2 Design Pattern \\ \hline
  State & Design Pattern \\ \hline
  Template Method & Design Pattern \\ \hline
  MVC & Architectural pattern \\ \hline
  Abstract Factory & Design Pattern \\ \hline
  Pipe and Filter & Deisgn Pattern \\ \hline
\end{tabular}
\\

\emph{3.b) How is the pattern you chose realized in your code? (Which class(es)
 works as the pattern you chose?)}
\\
We chose to implement the MVC-pattern in our assignment. In our implementation, 
the Ball and Paddle-class works as the model, and the State works as the View- Controller. 
Whenever the controllers TouchListener recieves a TouchEvent and needs to move the sprite, it’s call the 
setPosition-method of the appropriate Paddle-instance. To showcase a seperate View without controller attributes, we made a GameOverView for when the game is over.\\

\emph{3.c) Is there any advantages in using this pattern in this program? (What are the advan- tages/disadvantages?)}\\
An advantage of using the MVC-pattern is that the setLabel-method is only run at changes in the X- and Y-coordinates.




\bibliographystyle{plain}
\bibliography{}
\end{document}
