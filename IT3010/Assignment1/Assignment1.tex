\documentclass[11pt, a4paper]{article}
\usepackage[utf8x]{inputenc}
\usepackage{ucs}
\usepackage{float}
\usepackage{array}
\usepackage{amsmath}
\usepackage{amssymb}
\usepackage{amsfonts}
\usepackage{latexsym}
\usepackage{graphicx}
\usepackage{caption}
\usepackage{ifpdf}
\usepackage{url}
\usepackage{xtab}
\usepackage{geometry}
\usepackage{longtable}
\usepackage[hidelinks]{hyperref}
\usepackage[english]{babel} 
\usepackage[parfill]{parskip}
\usepackage{alltt}
\newcommand{\comment}[1]{} \comment{This is a block comment wrapped in curly
brackets}
%\renewcommand{\thefootnote}{\roman{footnote}}
\title{Research Proposal IT3010\\ Security surrounding electronic patient records} 
\author{Linn Vikre}
\date{\today}
\begin{document}
\maketitle
% Insert path to main sections here, using syntax \input{folder/filename}
\section{Motivation}
From the very beginning of the modern healthcare information system the security has been considered an important issue. This because of the highly sensitive personal data\cite{personopplysningsloven201104} that they contains, and it raises a concern around the guarantee of protection, confidentiality and integrity of patient information\cite{Smith199939}. 

It is crucial to be able to access the right information at the right time in a secure way in healthcare, this to prevent the wrong treatment of a patient\cite{Barber199819}. It is beneficial that several institutions may be linked to the same system so you can quickly have access to patient information when you need it. As of today in Norway the different institutions have their own health care systems to keep records due to that it does not exists a common system that is easy to use and secure enough. 

Further on are the data used to research and it is therefore important that the information is correct and that it only can be accessed by authorized people. This imposes a requirement for those who processes the data that they has the competence and addresses the rules in relation to restrictions on the use of patient data\cite{faxvaag2007er}.

The main aim of this research is to collect information about todays systems for processing patient records, identify which security issues that follows the use of patient data in healthcare information systems and suggest possible solutions.

\section{Relevant studies}

There exists multiple research papers that have looked into solving the security issues around keeping patient records safe in an electronic healthcare information system. The main focus on the research papers are both on the information system and the people that uses the systems e.g. doctors and nurses.

As described in a paper from 2003 by Brandner et al.\cite{van2003data} they have “Tumors Centers” in Germany that supports cross-institutional care processes. To obtain a secure and functional cross-institutional system, a detailed risk analysis that contained potential security issues that could occur in use of an electronic patient record system was required.

The paper describes possible measures and requirements that must, or at least should, be fulfilled by the health care system to meet the data protection agreement written in the EU Directive. Besides the technical requirements to preserve the security of personal data in a health care system, it was found that teaching the users of the system in data security and data protection are as important as the actual information system.

In the paper from 1999 by Smith and Eloff\cite{Smith199939} they describes the current trends in the security aspects of healthcare information systems. It discusses the dilemma of obtaining, using and sharing healthcare information without breaching patient privacy, and describes it as a serious concern as of today. 

The following conclusion of the paper is that trust in human professionals is the most critical factor when preserving the confidentiality in electronic patient records. A suggestion that contains standard guidelines and uniform policies concerning patient privacy is highly recommended. This also contains a secure information system, not only the trust of the users that uses the electronic patient record system.  

As of the two papers, it clearly states that it is important with a highly secured electronic patient record system to preserve the security of personal data. But it is just as important to have well trained employees that have been taught how to use the healthcare information system. These two papers are relevant for my study because they look at/ identify and discuss the main issues regarding an electronic patient record system and suggest some solutions that can be helpful in developing a system that preserves patient privacy in the future.

\section{Research questions and objectives}

From the previous information presented, this research will look at the security issues in electronic patient records and the requirements that should met in order to preserve patient privacy e.g. according to the EU Directive. It is a crucial study that can make it easier in the future to e.g share patient records across private- and government own institutions in a more secure way and save lives.   

\begin{itemize}
\item What are the requirements for information security to ensure confidentiality in patient records?
\item What are the main issues regarding security around electronic patient records in a healthcare information system?
\end{itemize}
\bibliographystyle{plain}
\bibliography{bibl}
\end{document}
