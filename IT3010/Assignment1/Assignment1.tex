\documentclass[11pt, a4paper]{article}
\usepackage[utf8x]{inputenc}
\usepackage{ucs}
\usepackage{float}
\usepackage{array}
\usepackage{amsmath}
\usepackage{amssymb}
\usepackage{amsfonts}
\usepackage{latexsym}
\usepackage{graphicx}
\usepackage{caption}
\usepackage{ifpdf}
\usepackage{url}
\usepackage{xtab}
\usepackage{geometry}
\usepackage{longtable}
\usepackage[hidelinks]{hyperref}
\usepackage[english]{babel} 
\usepackage[parfill]{parskip}
\usepackage{alltt}
\newcommand{\comment}[1]{} \comment{This is a block comment wrapped in curly
brackets}
%\renewcommand{\thefootnote}{\roman{footnote}}
\title{Research Proposal IT3010\\ Security surrounding electronic patient records in cross-institutional healthcare} 
\author{Linn Vikre}
\date{\today}
\begin{document}
\maketitle
% Insert path to main sections here, using syntax \input{folder/filename}
\section{Motivation}
From the very beginning of the modern healthcare information system the security has been considered an important issue. This because of the highly sensitive personal data that they contains, and it raises a concern around the guarantee of protection, confidentiality and integrity of patient information\cite{Smith199939}. In Norway today, the different institutions have their own health care systems to keep records due to that it does not exists a common system that is easy to use and secure enough.

It is crucial to be able to access the right information to the right time in a secure way in healthcare, this to prevent the wrong treatment of a patient. Further on are the data used to research and it is therefore important that the information is correct and that it only can be accessed by authorized people. This imposes a requirement for those who processes the data that they has the competence and addresses the rules in relation to restrictions on the use of patient data\cite{faxvaag2007er}. 

The main aim of this research is to collect information about todays systems for processing of patient data and identify which security issues that follows the use of patient data in healthcare information systems.

\section{Relevant studies}

There exists multiple research papers that have looked into solving the security issues around keeping patient records safe in an electronic healthcare information system. The main focus on the research papers are both on the information system and on the people that uses the systems e.g. doctors and nurses. 

\section{Research questions and objectives}

From the previous information presented, this research will look at the security issues in electronic patient records. It is a crucial study that can make it easier in the future to share patient records across private- and government own institutions and save lives. 

\begin{itemize}
\item What are the requirements for information security to ensure confidentiality in patient records?
\item What are the main issues regarding security around patient records as of today?
\end{itemize}
\bibliographystyle{plain}
\bibliography{bibl}
\end{document}
